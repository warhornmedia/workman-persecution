% Options for packages loaded elsewhere
\PassOptionsToPackage{unicode}{hyperref}
\PassOptionsToPackage{hyphens}{url}
%
\documentclass[
]{book}
\usepackage{lmodern}
\usepackage{amssymb,amsmath}
\usepackage{ifxetex,ifluatex}
\ifnum 0\ifxetex 1\fi\ifluatex 1\fi=0 % if pdftex
  \usepackage[T1]{fontenc}
  \usepackage[utf8]{inputenc}
  \usepackage{textcomp} % provide euro and other symbols
\else % if luatex or xetex
  \usepackage{unicode-math}
  \defaultfontfeatures{Scale=MatchLowercase}
  \defaultfontfeatures[\rmfamily]{Ligatures=TeX,Scale=1}
\fi
% Use upquote if available, for straight quotes in verbatim environments
\IfFileExists{upquote.sty}{\usepackage{upquote}}{}
\IfFileExists{microtype.sty}{% use microtype if available
  \usepackage[]{microtype}
  \UseMicrotypeSet[protrusion]{basicmath} % disable protrusion for tt fonts
}{}
\makeatletter
\@ifundefined{KOMAClassName}{% if non-KOMA class
  \IfFileExists{parskip.sty}{%
    \usepackage{parskip}
  }{% else
    \setlength{\parindent}{0pt}
    \setlength{\parskip}{6pt plus 2pt minus 1pt}}
}{% if KOMA class
  \KOMAoptions{parskip=half}}
\makeatother
\usepackage{fancyvrb}
\usepackage{xcolor}
\IfFileExists{xurl.sty}{\usepackage{xurl}}{} % add URL line breaks if available
\IfFileExists{bookmark.sty}{\usepackage{bookmark}}{\usepackage{hyperref}}
\hypersetup{
  pdftitle={Book Template},
  pdfauthor={Warhorn Classics},
  hidelinks,
  pdfcreator={LaTeX via pandoc}}
\urlstyle{same} % disable monospaced font for URLs
\VerbatimFootnotes % allow verbatim text in footnotes
\usepackage{longtable,booktabs}
% Correct order of tables after \paragraph or \subparagraph
\usepackage{etoolbox}
\makeatletter
\patchcmd\longtable{\par}{\if@noskipsec\mbox{}\fi\par}{}{}
\makeatother
% Allow footnotes in longtable head/foot
\IfFileExists{footnotehyper.sty}{\usepackage{footnotehyper}}{\usepackage{footnote}}
\makesavenoteenv{longtable}
\usepackage{graphicx}
\makeatletter
\def\maxwidth{\ifdim\Gin@nat@width>\linewidth\linewidth\else\Gin@nat@width\fi}
\def\maxheight{\ifdim\Gin@nat@height>\textheight\textheight\else\Gin@nat@height\fi}
\makeatother
% Scale images if necessary, so that they will not overflow the page
% margins by default, and it is still possible to overwrite the defaults
% using explicit options in \includegraphics[width, height, ...]{}
\setkeys{Gin}{width=\maxwidth,height=\maxheight,keepaspectratio}
% Set default figure placement to htbp
\makeatletter
\def\fps@figure{htbp}
\makeatother
\setlength{\emergencystretch}{3em} % prevent overfull lines
\providecommand{\tightlist}{%
  \setlength{\itemsep}{0pt}\setlength{\parskip}{0pt}}
\setcounter{secnumdepth}{5}
% DEFINE PHYSICAL DOCUMENT SETTINGS HD
% media settings
\usepackage[paperwidth=5.5in, paperheight=8in]{geometry}

\usepackage{booktabs}
\usepackage{amsthm}
\makeatletter
\def\thm@space@setup{%
  \thm@preskip=8pt plus 2pt minus 4pt
  \thm@postskip=\thm@preskip
}

\usepackage{titling}
\usepackage{pdfpages}
\IfFileExists{./cover.pdf}{
  \newcommand{\myCover}{./cover.pdf}}
  {\IfFileExists{./cover.jpg}{
    \newcommand{\myCover}{./cover.jpg}}
    {\IfFileExists{./cover.png}{
      \newcommand{\myCover}{./cover.png}}{}
    }
  }
\@ifundefined{myCover}
{}
{
\pretitle{\begin{center}\includepdf{\myCover}}
\posttitle{\end{center}\setcounter{page}{0}}
\usepackage{atbegshi}% http://ctan.org/pkg/atbegshi
\AtBeginDocument{\AtBeginShipoutNext{\AtBeginShipoutDiscard}}
}
\clearpage\pagenumbering{roman}

\newenvironment{poetry}[0]{\par\leftskip=2em\rightskip=2em}{\par\medskip}

\newfontfamily\greekfont[Script=Greek]{LiberationSerif}

\makeatother

\frontmatter
\ifluatex
  \usepackage{selnolig}  % disable illegal ligatures
\fi
\usepackage[]{natbib}
\bibliographystyle{plainnat}

\title{Book Template}
\author{Warhorn Classics}
\date{2020}

\begin{document}
\maketitle

\mainmatter
\pagenumbering{roman}

{
\setcounter{tocdepth}{1}
\tableofcontents
}
\hypertarget{about-this-book}{%
\chapter*{About this book}\label{about-this-book}}
\addcontentsline{toc}{chapter}{About this book}

Republished by \href{https://classics.warhornmedia.com/}{Warhorn Classics}---making classic Christian content available for free online in high quality, readable formats.

The latest version of this book can always be found \href{https://warhornmedia.github.io/warhorn-classics-book-template/}{here} in many electronic formats for your reading convenience on any device.

\hypertarget{downloads}{%
\subsubsection*{Downloads}\label{downloads}}
\addcontentsline{toc}{subsubsection}{Downloads}

\href{https://warhornmedia.github.io/warhorn-classics-book-template//Warhorn-Classics_Book_Template.pdf}{Download PDF}

\href{https://warhornmedia.github.io/warhorn-classics-book-template//Warhorn-Classics_Book_Template.epub}{Download ePub}

We hope this book is a blessing to you. If it is, please \href{https://warhornmedia.com/give}{make a one-time or recurring contribution} right now, sponsor a book from our upcoming list, or volunteer your proofreading or technical skills to help produce more content. Contact \href{mailto:lucas@beggarsborn.com}{Lucas Weeks} to get involved.

God bless,

---The Warhorn Team

\clearpage
\setcounter{page}{1}\pagenumbering{arabic}

\hypertarget{basic-instructions}{%
\chapter{Basic instructions}\label{basic-instructions}}

\hypertarget{creating-a-new-repo-for-a-new-book-on-github}{%
\section{Creating a new repo for a new book on Github}\label{creating-a-new-repo-for-a-new-book-on-github}}

\begin{enumerate}
\def\labelenumi{\arabic{enumi}.}
\tightlist
\item
  Go to \href{https://github.com/warhornmedia/warhorn-classics-book-template}{the repo} on Github and click ``Use this template.''
\end{enumerate}

\begin{center}\includegraphics[width=0.65\linewidth]{images/screenshot1} \end{center}

\begin{enumerate}
\def\labelenumi{\arabic{enumi}.}
\setcounter{enumi}{1}
\tightlist
\item
  Change the owner to warhornmedia. Enter a repository name using the format ``authorlastname-short-book-title''. Set the repository to public. And include all branches. Then click ``Create repository from template''
\end{enumerate}

\begin{center}\includegraphics[width=0.65\linewidth]{images/screenshot2} \end{center}

\begin{enumerate}
\def\labelenumi{\arabic{enumi}.}
\setcounter{enumi}{2}
\item
  Create an account on travis-ci.com and link it to your github account. Make sure to give it permission for the warhornmedia organization as well.
\item
  Clone the new repo to your computer and navigate into its folder in Terminal. Then run the following command locally, answering `yes' to both questions.\footnote{\href{https://docs.github.com/en/free-pro-team@latest/github/authenticating-to-github/creating-a-personal-access-token}{Here are instructions} for creating a Github Personal Access Token if you don't have one yet. (Check the ``repo'' box under ``scopes'' to give the necessary permissions.)

    Also, if you don't have \href{https://github.com/travis-ci/travis.rb\#installation}{travis} installed on your computer yet, you can install it with this command:

\begin{Verbatim}
  brew install travis
\end{Verbatim}

    Before this will work, you will need to create an account on travis-ci.com and link it to your github account.

    If you don't have \href{https://brew.sh}{brew} installed yet\ldots{} prepare yourself for some waiting. You can install it with this command:

\begin{Verbatim}
  /bin/bash -c "$(curl -fsSL https://raw.githubusercontent.com/Homebrew/install/master/install.sh)"
\end{Verbatim}
  }

\begin{verbatim}
travis encrypt GITHUB_PAT=yourTokenGoesHere --com --add -x
\end{verbatim}
\end{enumerate}

This will encrypt your token so that only Travis CI can use it. Then it will add it to the end of the .travis.yml file. Because it is encrypted, it doesn't matter that it is available in our public repo in the .travis.yml file. In fact, it has to be there. Without that, Travis CI wouldn't be able to save the updated book files back to Github.

\hypertarget{preparing-your-local-build-environment}{%
\section{Preparing your local build environment}\label{preparing-your-local-build-environment}}

\begin{enumerate}
\def\labelenumi{\arabic{enumi}.}
\tightlist
\item
  Download RStudio for desktop \href{https://rstudio.com/products/rstudio/}{here}.
\item
  Make sure that \href{https://www.fontsquirrel.com/fonts/liberation-serif}{LiberationSerif} is installed.
\end{enumerate}

\emph{Congrats!} You now have a new book that will rebuild automatically any time you push changes to github.

For more in-depth instructions on setting up your new book, as well as important information on how to code the book, check out the \href{https://warhornmedia.github.io/style-guide}{style guide}.

\end{document}
